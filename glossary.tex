\newglossaryentry{SalesEnablement}
{
    name=Sales Enablement,
    description={Sales enablement is the process of providing the sales organization with the information, content and tools that help salespeople sell more effectively. }
}
\newglossaryentry{CI/CD}
{
    name=CI/CD,
    description={Continuous Integration/Continuous Delivery is a process framework Mathematics is what mathematicians do}
}
\newglossaryentry{Scrum}
{
    name=Scrum,
    description={Scrum is one of the most popular agile methodologies, based on a timeboxed iteration (sprint), with a clearly defined goal and scope. It comes with a limited set of roles and ceremonies to stay as light weight as possible.}
}
\newglossaryentry{Epic}
{
    name=Epic,
    description={At Showpad, an Epic is the unit of Customer Value delivery. The combination of all Epics forms the Roadmap. To enable a value delivery focus, we limit the scope of an Epic to typically fit in one or a maximum of two release cycles. Epics are large bodies of work that can be broken down into a number of User Stories. In an Epic, we focus on tackling 1 and only 1, clearly defined problem specific to 1 principal Actor’s (user persona) User Experience. Multiple teams can work on the same Epic. The Epic is to a Release, what a User Story is to a Sprint.
    }
}
\newglossaryentry{Userstory}
{
    name=User Story,
    description={At Showpad, User Stories are short requirements or requests written from the perspective of an end-user. A User Story fits into the timebox of a Sprint and is the lowest level of granularity for planning and delivering work. A User Story has a 1-on-1 relationship with a team that will implement it.}
}
\newglossaryentry{Sprint}
{
        name=Sprint,
        description={A sprint is a short, time-boxed period when a scrum team works to complete a set amount of work. At Showpad, a sprint is typically 2 to 3 weeks long, it is up to the Scrum team to determine the length of Sprints. once chosen, it puts the team in a fixed cadence of delivery.}
}
\newglossaryentry{Monorepo}
{
    name=Monorepo,
    description={Two dominant models for structuring code bases exist in the version control space. Monorepo and Polyrepo. In a Monorepo setup, the codebases of the multiple projects are kept together in a single repository. These projects have a well-defined relationship. The colocation of code and the well-defined relationship among the projects make a Monorepo a Monorepo.}
}
\newglossaryentry{Cycletime}
{
    name=Cycle Time,
    description={Cycle Time is the elapsed time from the start of the first task to finishing the last job to deliver 1 work item.}
}

\newacronym{CI}{CI}{Continuous Integration}
\newacronym{CD}{CD}{Continuous Delivery}
\newacronym{SAAS}{SAAS}{Software as a Service}
\newacronym{ARR}{ARR}{Annual Recurring Revenue}
\newacronym{CRM}{CRM}{Customer Relationship Management}
\newacronym{DDD}{DDD}{Domain Driven Design}

    